% OHILab C/C++ Coding Style
% (C) 2012-2018 - OHILab <info@ohilab.org>
%
% Author: 
%	Marco Giammarini <m.giammarini@warcomeb.it>

\section{Commenti}\label{sec:comments}

In questa sezione viene affrontata una delle parti più importanti, e forse più critica, della programmazione: la documentazione del codice.
La documentazione è fondamentale per permettere ad altri sviluppatori di poter interagire, modificare o ampliare il codice senza doverne analizzare ogni singola riga per capirne il funzionamento. 
Inoltre documentare il codice permette anche, nel caso si documenti un'applicazione, di avere pronto un buon manuale d'uso.
L'\sigla ha scelto Doxygen\cite{codestyle:doxygen} come strumento per la generazione della documentazione di librerie e applicazioni.

% \begin{minipage}[t]{\rbwidth}
% \begin{RightSmallCode}
% /******************************************************************************
%  * Copyright (C) 2012-2013 A. C. Open Hardware Ideas Lab
%  *
%  * Author:
%  *   Name Surname <email@domain.com>
%  *
%  * Project: @project_name@
%  * Package: ...
%  * Version: @project_version@
%  *
%  * TODO: LICENZA
%  *
%  * THE SOFTWARE IS PROVIDED "AS IS", WITHOUT WARRANTY OF ANY KIND,
%  * EXPRESS OR IMPLIED, INCLUDING BUT NOT LIMITED TO THE WARRANTIES OF
%  * MERCHANTABILITY, FITNESS FOR A PARTICULAR PURPOSE AND
%  * NONINFRINGEMENT. IN NO EVENT SHALL THE AUTHORS OR COPYRIGHT HOLDERS
%  * BE LIABLE FOR ANY CLAIM, DAMAGES OR OTHER LIABILITY, WHETHER IN AN
%  * ACTION OF CONTRACT, TORT OR OTHERWISE, ARISING FROM, OUT OF OR IN
%  * CONNECTION WITH THE SOFTWARE OR THE USE OR OTHER DEALINGS IN THE
%  * SOFTWARE.
%  *
%  *****************************************************************************/
% \end{RightSmallCode}
% \end{minipage}%
