% OHILab C/C++ Coding Style
% (C) 2012-2018 - OHILab <info@ohilab.org>
%
% Author: 
%	Marco Giammarini <m.giammarini@warcomeb.it>
%	Vito Colagiacomo <vito.colagiacomo@gmail.com>

\section{Introduzione}\label{sec:overview}

In questo guida sono riportate le convenzioni utilizzate dell'\associazione per la scrittura di codice nei linguaggi C e C++.
Le regole espresse in questo documento devono essere seguite attentamente e fedelmente da ogni sviluppatore che si accinge a scrivere codice che dovrà poi essere integrato in un qualsiasi progetto dell'\sigla.
L'utilizzo delle regole che vengono riportate, oltre a permettere una maggiore coerenza del codice, ne permetterà anche una migliore lettura e comprensione da parte di altri sviluppatori.

Per la stesura di questa guida si è fatto riferimento principalmente a due documenti: \emph{Google C++ Style Guide}\cite{codestyle:google} ed al \emph{GNU Coding Standards}\cite{codestyle:gnu} dai quali si è cercato di estrarre un set di regole basi comuni sia al buon senso ma anche al mondo del software in generale.

Per una maggiore comprensione delle regole nella guida verranno presentati molti esempi sia di scrittura corretta, racchiusi in dei riquadri verdi, sia esempi di scrittura non corretta, racchiusi in dei riquadri rossi.

Ovviamente la guida non pretende di essere esaustiva e non copre tutti i particolari casi.
In caso di dubbi è bene affidarsi al buon senso avendo sempre come basamento le regole che verranno illustrate; nel caso vogliate avere chiarimenti maggiori o preferite confrontarvi con noi potete scrivere al referente del \emph{Development Team} al nostro indirizzo \url{info@ohilab.org}.
