% OHILab C/C++ Coding Style
% (C) 2012 - OHILab <info@ohilab.org>
%
% Author: 
%	Marco Giammarini <m.giammarini@warcomeb.it>
%	Vito Colagiacomo <vito.colagiacomo@gmail.com>
%
% svn:
%	$Id: name.tex 139 2012-12-03 08:05:19Z warcomeb $

\section{Nomi}\label{sec:name}

I nomi di variabili, funzioni, tipi di dato, file, etc. devono essere \emph{sempre} scelti in modo da fornire anche un'informazione riguardo il loro significato.
È cattiva norma comune, infatti, usare abbreviazioni e codici strani per i nomi, con conseguente illeggibilità del codice da parte di altri sviluppatori e spesso anche dallo sviluppatore che lo ha scritto.
Per le variabili locali invece può essere scelto un nome più corto, poiché il loro utilizzo è limitato all'interno del loro \emph{scope}, dove in genere un commento descrive il loro utilizzo, ad ogni modo è bene cercare di limitare il numero di abbreviazioni. 

La nomenclatura di tutti gli elementi all'interno del codice deve seguire la notazione \emph{Camel Case}\cite{codestyle:camel} con le varianti che verranno mostrate nei prossimi paragrafi.

La lingua adottata è l'inglese.

\subsection{Abbreviazioni ed acronimi}\label{ssec:abbreviations}

Come già detto in precedenza, è bene evitare di utilizzare nomi ambigui ed abbreviazioni di difficile comprensione:

\begin{minipage}[t]{\cbwidth}
\begin{RightCode}
int numElements;
int errorCount;
bool isSelected;
\end{RightCode}
\end{minipage}%
\hspace{\cbdistance}
\begin{minipage}[t]{\cbwidth}
\begin{ErrorCode}
int nElements;
int eCnt;
bool sel;
\end{ErrorCode}
\end{minipage}

Acronimi e abbreviazioni di uso comune devono essere usati ma non possono essere indicati interamente in maiuscolo, mentre non devono assolutamente essere usate le forme complete degli acronimi:

\begin{minipage}[t]{\cbwidth}
\begin{RightCode}
getJsonResponse();
setPwmDutyCicle();
getHtml();
\end{RightCode}
\end{minipage}%
\hspace{\cbdistance}
\begin{minipage}[t]{\cbwidth}
\begin{ErrorCode}
getJSONResponse();
setPWMDutyCicle();
getHypertextMarkupLanguage();
\end{ErrorCode}
\end{minipage}

Inoltre, generalmente è pratica comune utilizzare la stessa variabile locale (con nomi tipo \texttt{tmp}) per scopi differenti all'interno della stessa funzione.
Anziché fare ciò, è meglio utilizzare variabili locali differenti per ogni scopo assegnadogli nomi significativi.
Questo aiuta a comprendere meglio il programma, lasciando al compilatore il compito di ottimizzare il codice.
Se poi si cerca di posizionare la dichiarazione della varibile locale all'interno dello \emph{scope} più piccolo che comprende il suo utilizzo rende il codice ulteriormente più chiaro.

\subsection{File}\label{ssec:filename}

Per nomi dei file del progetto che contengono codice C/C++ è \emph{strettamente necessario} attenersi alle seguenti regole:
\begin{itemize}
	\item I nomi dei file devono essere scritti tutti in minuscolo e non possono contenere caratteri come lo spazio, l'underscore e il trattino.
	\item I file header del progetto devono avere estensione \texttt{.h}.
	\item I file sorgente del progetto devono avere estensione \texttt{.c} nel caso di codice C, \texttt{.cpp} nel caso di codice C++.
	\item È bene che i nomi dei file siano, come per ogni nome all'interno del progetto, il più chiaro e specifico possibile.
		Ad esempio non chiamare un file \texttt{logs.h} se questo gestisce il log di una specifica parte del progetto.
	\item Se il file contiene la dichiarazione o la definizione di una sola struttura di dati o classe, il suo nome deve essere uguale al nome della struttura ivi contenuta.
		Ad esempio se il file \texttt{.h} contiene la definizione della sola classe \texttt{PippoStar}, questo dovrà essere nominato \texttt{pippostar.h}, ed il relativo file contenente l'implementazione della classe \texttt{pippostar.cpp}. 
\end{itemize}

\subsection{Tipi di dato}\label{ssec:typename}

I nomi dei tipi di dato (classi, strutture, typedef ed enum) devono iniziare con la prima lettera maiuscola, ed hanno una lettera maiuscola per ogni nuova parola\cite{codestyle:camel}.

\begin{minipage}[t]{\cbwidth}
\begin{RightCode}
class SerialBus { ... };
enum SerialResponse { ... };
\end{RightCode}
\end{minipage}%
\hspace{\cbdistance}
\begin{minipage}[t]{\cbwidth}
\begin{ErrorCode}
class serialbus { ... };
enum Serial_Response { ... };
\end{ErrorCode}
\end{minipage}

\subsection{Variabili}\label{ssec:variable}

\subsubsection{Variabili ordinarie e membro di strutture}\label{sssec:varnorm}

Per distinguere il nome delle variabili da quello dei tipi di dato, nelle variabili la prima lettera deve essere minuscola.
Anche in questo caso il carattere underscore non deve essere mai utilizzato per separare le parole.
Inoltre, è buona norma utilizzare forme plurali nei nomi di variabili che rappresentano collezioni di oggetti.

\begin{minipage}[t]{\cbwidth}
\begin{RightCode}
int numElements;
int errorsCount;
int values[];
\end{RightCode}
\end{minipage}%
\hspace{\cbdistance}
\begin{minipage}[t]{\cbwidth}
\begin{ErrorCode}
int num_elements;
int ErrorsCount;
int value[];
\end{ErrorCode}
\end{minipage}

\subsubsection{Variabili membro di classi}

I nomi delle variabili membro di una classe seguono la stessa convenzione definita per le variabili ordinarie, anteponendo però al nome della variabile il carattere \emph{m}.
Questo per differenziare, all'interno delle funzioni membro della classe, la gestione della variabile membro con quelle passate come parametro alla funzione stessa.
%È vietato anteporre l'underscore al nome della variabile poiché é una convenzione spesso usata per nomi di parole chiave (i.e.: \texttt{\_\_declspec}).  

\begin{minipage}[t]{\cbwidth}
\begin{RightCode}
string mName;
int mBirthdayYear;
\end{RightCode}
\end{minipage}%
\hspace{\cbdistance}
\begin{minipage}[t]{\cbwidth}
\begin{ErrorCode}
string name_;
int _birthdayYear;
\end{ErrorCode}
\end{minipage}

\subsubsection{Variabili globali}

Per quanto riguarda le variabili globali queste seguono le stesse regole delle variabili ordinarie (paragrafo~\ref{sssec:varnorm}).
È buona norma inoltre riferirsi a queste variabili usando l'operatore di scope globale <<::>>\cite{codestyle:geotechnical}.

\begin{minipage}[t]{\rbwidth}
\begin{RightCode}
int a = ::myGlobalVariable;
\end{RightCode}
\end{minipage}

\subsubsection{Variabili contatore}

I nomi delle variabili contatore da utilizzare nei cicli, se il loro scope è molto limitato, dovrebbero essere preferibilmente \texttt{i}, \texttt{j}, \texttt{k}, \texttt{m}, \texttt{n}, altrimenti, se lo scope non è limitato, è bene usare un nome più esplicativo.

\subsection{Funzioni}\label{ssec:functionname}

Come per le variabili, anche le funzioni devono seguire la notazione \emph{Camel Case}, iniziando con la lettera minuscola\cite{codestyle:geotechnical}.
È buona norma che il nome di una funzione rappresenti un'azione e che quindi contenga il \emph{verbo} che la descrive.
Anche per il nome delle funzioni non è consentito l'uso dell'underscore.    

\begin{minipage}[t]{\cbwidth}
\begin{RightCode}
addTableEntry()
deleteUrl()
\end{RightCode}
\end{minipage}%
\hspace{\cbdistance}
\begin{minipage}[t]{\cbwidth}
\begin{ErrorCode}
AddTableEntry()
delete_url()
\end{ErrorCode}
\end{minipage}

I nomi delle funzioni utilizzate per accedere e modificare direttamente variabili membro di una classe devono iniziare rispettivamente con \texttt{get} e \texttt{set}, seguiti dal nome della variabile.

\begin{minipage}[t]{\rbwidth}
\begin{RightCode}
employee.getName();
employee.setName(name);
\end{RightCode}
\end{minipage}%

Una regola da ricordare nel dare nomi alle funzioni è che il nome dell'oggetto è implicito e quindi non deve essere ripetuto all'interno del nome della funzione.

\begin{minipage}[t]{\cbwidth}
\begin{RightCode}
line.getLength();
\end{RightCode}
\end{minipage}%
\hspace{\cbdistance}
\begin{minipage}[t]{\cbwidth}
\begin{ErrorCode}
line.getLineLength();
\end{ErrorCode}
\end{minipage}

\subsection{Namespace}\label{ssec:namespacename}

I nomi dei namespace devono essere composti solo da lettere minuscole e senza underscore tra le parole\cite{codestyle:geotechnical}.

\begin{minipage}[t]{\cbwidth}
\begin{RightCode}
device::serialport
\end{RightCode}
\end{minipage}%
\hspace{\cbdistance}
\begin{minipage}[t]{\cbwidth}
\begin{ErrorCode}
device::serial_port
\end{ErrorCode}
\end{minipage}

\subsection{Costanti}\label{ssec:constantname}

I nomi delle costanti, contrariamente alle altre variabili, devono essere scritte interamente con lettere maiuscole e con le parole separate da un underscore\cite{codestyle:geotechnical}.
Questa regola è valida anche per i valori degli \texttt{enum}.

\begin{minipage}[t]{\cbwidth}
\begin{RightCode}
const int MAX_DIMENSION = 10;
const int NUM_ITERATIONS = 2;
\end{RightCode}
\end{minipage}%
\hspace{\cbdistance}
\begin{minipage}[t]{\cbwidth}
\begin{ErrorCode}
const int maxDimension = 10;
const int NUMITERATIONS = 2;
\end{ErrorCode}
\end{minipage}

Per aumentare la coerenza e la leggibilità del codice, potrebbe essere considerata una buona norma la definizione di una funzione al posto di una costante\cite{codestyle:geotechnical}.
Questa soluzione \emph{non} deve essere usata in ambito embedded.

\begin{minipage}[t]{\rbwidth}
\begin{RightCode}
int
getNumIterations ()
{
	return 2;
}
\end{RightCode}
\end{minipage}

\subsection{Enumerators}\label{ssec:enumname}

I valori degli \texttt{enum}, come già detto, seguono le regole delle costanti.
Volendo, nell'ottica di aumentare la leggibilità del codice, è possibile aggiungere come prefisso ai nomi dei valori il nome del tipo:

\begin{minipage}[t]{\rbwidth}
\begin{RightCode}
enum Color
{
	COLOR_RED = 1,
	COLOR_GREEN,
	COLOR_BLUE
};
\end{RightCode}
\end{minipage}

\subsection{Macro}\label{ssec:macroname}

Le macro seguono le stesse regole indicate precedentemente per le variabili costanti e gli enumerators.

\subsection{Template}\label{ssec:templatename}

I nomi rappresentanti template devono essere composti da una sola lettera maiuscola.

\begin{minipage}[t]{\rbwidth}
\begin{RightCode}
template<class T> ...
template<class C, class D> ...
\end{RightCode}
\end{minipage}

\subsection{Termini, prefissi e suffissi}\label{ssec:prefixsuffix}

In questo paragrafo verranno riportati alcuni consigli e regole tratti da \cite{codestyle:geotechnical} sull'utilizzo di alcuni termini, suffissi e prefissi.
Si consiglia di seguire molto attentamente quanto riportato.% ed in particolari le regole con l'item rosso.

Il prefisso \emph{is} deve essere usato come prefisso per le variabili boolean, mentre non è consentito negare il nome della variabile.

\begin{minipage}[t]{\cbwidth}
\begin{RightCode}
bool isValid;
\end{RightCode}
\end{minipage}%
\hspace{\cbdistance}
\begin{minipage}[t]{\cbwidth}
\begin{ErrorCode}
bool isNoValid;
\end{ErrorCode}
\end{minipage}

Allo stesso modo le funzioni che ritornano variabili boolean dovrebbero avere questo prefisso.
Altre alternative sono i prefissi \emph{can}, \emph{has} e \emph{should}.

\begin{minipage}[t]{\rbwidth}
\begin{RightCode}
bool hasLicense ();
bool isUpdated ();
\end{RightCode}
\end{minipage}

Una buona norma per ridurre la complessità e diminuire il numero di nomi delle variabili consiste nel dare a variabili generiche lo stesso nome del tipo di dato.

\begin{minipage}[t]{\cbwidth}
\begin{RightCode}
setTopic (Topic* topic);
connect (Database* database);
\end{RightCode}
\end{minipage}%
\hspace{\cbdistance}
\begin{minipage}[t]{\cbwidth}
\begin{ErrorCode}
setTopic (Topic* value);
connect (Database* db);
\end{ErrorCode}
\end{minipage}

Variabili che invece hanno un ruolo e che quindi non sono generiche devono avere nomi significativi.

\begin{minipage}[t]{\rbwidth}
\begin{RightCode}
Point startingPoint, centerPoint;
Name loginName;
\end{RightCode}
\end{minipage}%


Il prefisso \emph{n} può essere usato per variabili che indicano una quantità.

\begin{minipage}[t]{\rbwidth}
\begin{RightCode}
int nLines;
int nPoints;
\end{RightCode}
\end{minipage}

Il suffisso \emph{No} può essere usato per variabili che indicano il numero di un particolare oggetto.

\begin{minipage}[t]{\rbwidth}
\begin{RightCode}
int linesNo;
int pointsNo;
\end{RightCode}
\end{minipage}

Il termine \emph{compute} deve essere usato dove qualcosa viene effettivamente calcolato.

\begin{minipage}[t]{\rbwidth}
\begin{RightCode}
matrix->computeInverse ();
\end{RightCode}
\end{minipage} 

Il termine \emph{find} deve essere usato dove qualcosa viene effettivamente cercato.

\begin{minipage}[t]{\rbwidth}
\begin{RightCode}
vector->findMinElement ();
\end{RightCode}
\end{minipage} 

Il termine \emph{initialize} deve essere usato quando viene inizializzato un oggetto o una procedura.

\begin{minipage}[t]{\rbwidth}
\begin{RightCode}
interface.initializeUsersTree ();
\end{RightCode}
\end{minipage}

Per ridurre la complessità di lettura del codice è buona norma dare nomi complementari a funzioni che eseguono operazioni complementari.
Vengono di seguito riportati alcuni esempi:
\begin{itemize}
	\item \texttt{set}/\texttt{get},
	\item \texttt{add}/\texttt{remove},
	\item \texttt{create}/\texttt{destroy}, 
	\item \texttt{start}/\texttt{stop},
	\item \texttt{insert}/\texttt{delete},
	\item \texttt{increment}/\texttt{decrement},
	\item \texttt{old}/\texttt{new},
	\item \texttt{begin}/\texttt{end},
	\item \texttt{first}/\texttt{last},
	\item \texttt{up}/\texttt{down},
	\item \texttt{min}/\texttt{max},
	\item \texttt{next}/\texttt{previous},
	\item \texttt{open}/\texttt{close},
	\item \texttt{show}/\texttt{hide},
	\item \texttt{suspend}/\texttt{resume},
	\item etc\dots
\end{itemize}

Classi o strutture rappresentanti eccezioni devono avere il suffisso \emph{Exception}.

\begin{minipage}[t]{\rbwidth}
\begin{RightCode}
class IOException
{
	...
};
\end{RightCode}
\end{minipage}%

\subsection{Moduli embedded}

Un discorso leggermente diverso va fatto quando ci si accinge a scrivere firmware e si intende rendere modulare il codice e quindi creare un blocco per ogni periferica presente nella scheda e nel microcontrollore.
Tutte le regole mostrate fino ad ora continuano a valere, ma con qualche piccolo accorgimento e modifica.

Si prenda ad esempio la periferica seriale di un qualsiasi microcontrollore.
All'interno dei file che descriveranno le funzioni da poter utilizzare sarà presente, ad esempio, la struttura \texttt{Serial} che verrà utilizzata per memorizzare i dati ed i parametri di una relativa seriale (il microcontrollore ne potrebbe avere più di una).
Per fare in modo che non ci siano rischi di funzioni e variabili globali che si chiamino uguali all'interno di più moduli, ognuna di queste dovrà avere come prefisso il nome del modulo seguito dal carattere underscore.

\begin{minipage}[t]{\rbwidth}
\begin{RightCode}

int Serial_maxBufferDimension = 10;

void Serial_sendChar(char);
char Serial_readChar();
\end{RightCode}
\end{minipage}

Nella sezione~\ref{sec:firmware} verrà dettagliata meglio la modalità di scrittura di un modulo per la gestione delle periferiche di un microcontrollore.