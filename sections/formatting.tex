% OHILab C/C++ Coding Style
% (C) 2012 - OHILab <info@ohilab.org>
%
% Author: 
%	Marco Giammarini <m.giammarini@warcomeb.it>
%	Vito Colagiacomo <vito.colagiacomo@gmail.com>
%
% svn:
%	$Id: formatting.tex 140 2012-12-04 08:24:12Z warcomeb $

\section{Formattazione}\label{sec:formatting}

In questa sezione verranno illustrate le principali regole di formattazione che dovranno essere seguite per la stesura del codice.

La prima regola fondamentale da tenere in mente e che viene riportata da quasi tutte le guide di coding style\cite{codestyle:google,codestyle:geotechnical,codestyle:quantum} è che ogni riga di codice \emph{non} deve superare le 80 colonne.

\subsection{Spazi e tabulazioni}

Per indentare il codice è permesso il solo uso degli spazi, in particolare l'indentazione è di \emph{quattro} spazi.
L'uso del carattere di tabulazione è assolutamente vietato.
Il motivo di tale scelta risiede nel fatto che si vuole evitare problemi di codifica tra diversi IDE e diversi sistemi operativi.
Inoltre gli spazi non hanno interpretazioni diverse, in temini di spazio occupato, in base alle impostazioni dell'IDE, cosa che invece succede per la tabulazione.
Ogni IDE permette di configurare la generazione del numero di caratteri richiesti quando viene premuto il tasto \emph{TAB}.

\subsection{Classi, strutture ed enum}

% TODO: da fare!!!

\subsection{Dichiarazione e definizione di funzioni}

La dichiarazione e la definizione di una funziona deve essere possibilmente fatta in un'unica riga.
Nella definizione il tipo di dato di ritorno va posizionato su una riga e il nome della funzione e i suoi parametri all'inizio della riga successiva.
Questo aiuta gli altri sviluppatori a trovare dove comincia una nuova funzione.

Le parentesi graffe che contengono l'implementazione della funzione devono essere posizionate nella prima colonna di una nuova riga di codice.
Infatti molti tool utili a indicizzare le funzioni presenti in un file cercano proprio nella prima colonna la parentesi graffa di apertura per localizzare la loro implementazione.

\begin{minipage}[t]{\rbwidth}
\begin{RightCode}
ReturnType
functionName (Type1 argument1, Type2 argument2)
{
  ...
}
\end{RightCode}
\end{minipage}%

Nel caso in cui gli argomenti della funzione siano piuttosto numerosi, i parametri possono essere suddivisi su più righe, mantenendo comunque valido il limite del numero di colonne.
Quando parte dei parametri vengono messi su di una nuova linea e bene che siano allineati con quelli della riga sopra.

\begin{minipage}[t]{\rbwidth}
\begin{RightCode}
ReturnType
longFunctionName (Type1 argument1, Type2 argument2,
                  Type3 argument3)
{
  ...
}

ReturnType
otherLongFunctionName (Type1 argument1,
                       Type2 argument2,
                       Type3 argument3)
{
  ...
}
\end{RightCode}
\end{minipage}%

Inoltre è bene tenere a mente alcune regole:
\begin{itemize}
	\item come già detto, le parentesi graffe aperte e chiuse vanno messe da sole nella prima colonna;
	\item uno spazio va inserito tra il nome della funzione e la parentesi tonda aperta;
	\item il nome dei parametri va indicato sia nella dichiarazione che nella definizione;
	\item se la funzione è \texttt{const}, la relativa keyword va inserita sulla stessa riga dell'ultimo parametro dopo la parentesi tonda;
	\item non ci sono spazi tra le parentesi tonde e i parametri.
\end{itemize}

\subsection{Chiamata di funzioni}

Come per la definizione e la dichiarazione di una funzione, anche la chiamata della funzione va posizionata, compatibilmente con la sua lunghezza, su di un'unica riga.
Nel caso non fosse possibile, le regole da seguire sono le stesse usate per la definizione.

\begin{minipage}[t]{\rbwidth}
\begin{RightCode}
value = functionName(argument1, argument2);

valueLong = longFunctionName(argument1, argument2, argument3,
                             longArgumentName);

valueLongLong = VeryLongFunctionName(argument1,
                                     argument2,
                                     argument3,
                                     longArgumentName);
\end{RightCode}
\end{minipage}%

Si noti come in questo caso non sia presente lo spazio tra il nome della funzione e la parentesi tonda aperta.

\subsection{Costrutto \texttt{if}}

Il costrutto \texttt{if} segue le stesse regole della definizione delle funzioni e deve avere la seguente forma.

\begin{minipage}[t]{\rbwidth}
\begin{RightCode}
if (condition)
{
  ...
}
else
{
  ...
}
\end{RightCode}
\end{minipage}%

Si noti lo spazio presente tra il nome del costrutto e la parentesi tonda contenente la condizione.
È possibile scrivere l'istruzione condizionale, nel caso sia di breve lunghezza, sulla stessa riga del costrutto nel caso in cui noi sia presente anche l'\texttt{else}.

\begin{minipage}[t]{\cbwidth}
\begin{RightCode}
if (x == kFoo) doThis();
\end{RightCode}
\end{minipage}%
\hspace{\cbdistance}
\begin{minipage}[t]{\cbwidth}
\begin{ErrorCode}
if (x) doThis();
else doThat();
\end{ErrorCode}
\end{minipage}

Nel caso di istruzioni condizionali formate da una singola riga, non è necessario inserire le parentesi graffe.
Nel caso in cui una parte del costrutto \texttt{if-else} le usi, allora anche l'altro deve utilizzarle, anche se solo per una riga.

\begin{minipage}[t]{\cbwidth}
\begin{RightCode}
if (condition)
{
  foo;
}
else
{
  bar;
}
\end{RightCode}
\end{minipage}%
\hspace{\cbdistance}
\begin{minipage}[t]{\cbwidth}
\begin{ErrorCode}
if (condition)
{
  foo;
}
else
  bar;
\end{ErrorCode}
\end{minipage}

\subsection{Costrutti \texttt{for} e \texttt{while}}

I costrutti \texttt{for} e \texttt{while} seguono la stessa sintassi indicata per il costrutto \texttt{if}.

\begin{minipage}[t]{\rbwidth}
\begin{RightCode}
for (initialization; condition; update)
{
  statement;
}

while (condition)
{
  statement;
}
\end{RightCode}
\end{minipage}%

Una particolarità a cui bisogna prestare attenzione sono i cicli con corpi vuoti.
Questa tipologia di ciclo deve sempre prevedere le parentesi o l'istruzione \texttt{continue}.

\begin{minipage}[t]{\cbwidth}
\begin{RightCode}
while (condition) {}
while (condition) continue;
\end{RightCode}
\end{minipage}%
\hspace{\cbdistance}
\begin{minipage}[t]{\cbwidth}
\begin{ErrorCode}
while (condition) ;
\end{ErrorCode}
\end{minipage}

\subsection{Costrutto \texttt{switch}}

Il costrutto \texttt{switch} viene scritto in molti modi e tipologie diverse.
In questa guida si tiene in considerazione che ogni \texttt{case} rappresenta una label e che quindi non va indentata.
Il costrutto \texttt{switch} dovrebbe sempre prevedere il caso \texttt{default}.

\begin{minipage}[t]{\rbwidth}
\begin{RightCode}
switch (var)
{
case 0:   // 0 space indent
    ...   // 4 space indent
    break;
case 1:
    ...
    break;
default:
    ...
}
\end{RightCode}
\end{minipage}%