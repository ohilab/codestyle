% OHILab C/C++ Coding Style
% (C) 2012-2018 - OHILab <info@ohilab.org>
%
% Author: 
%	Marco Giammarini <m.giammarini@warcomeb.it>

\section{Nomi nel Firmware}\label{sec:firmware}

Un discorso leggermente diverso va fatto quando ci si accinge a scrivere firmware e si intende rendere modulare il codice e quindi creare un blocco per ogni periferica presente nella scheda e nel microcontrollore.
Il principio di base è che il fimrware deve essere scritto in maniera modulare provando a pensare il firmware come fosse un'applicazione scritta con un linguaggio orientato agli oggetti.
Fondamentalmente, ogni tipo di periferica del microcontrollore è come se fosse una classe, ed ogni periferica presente al suo interno è come se fosse un'istanza della relativa classe.
Lo stesso discorso deve valere per ogni componente (intelligente ovviamente) presente sulla scheda elettronica.

Ricapitolando, tutte le regole mostrate fino ad ora continuano a valere, ma con qualche piccolo accorgimento e modifica che verranno mostrate di seguito.
Dove non indicato, valgono le regole presentate nella sezione~\ref{sec:name}.
Tutti gli esempi che verranno mostrati prendono ad esempio la periferica seriale di un qualsiasi microcontrollore, il cui modulo verrà chiamato a titolo di esempio \texttt{Serial}.

\subsection{Tipi di dato}\label{ssec:etypename}

I nomi dei tipi di dato (strutture, typedef ed enum) devono iniziare con la prima lettera maiuscola, ed hanno una lettera maiuscola per ogni nuova parola\cite{codestyle:camel}.
Nel caso ci si trovi all'interno di un modulo, il nome di quest'ultimo deve precedere sempre il nome del tipo di dato e diviso da questo mediante l'underscore.
Il nome del tipo di dato deve comunque iniziare con la lettera maiuscola.

\noindent\begin{minipage}[t]{\cbwidth}
\begin{RightCode}
struct Serial_Device { ... };
enum Serial_StartBit { ... };
\end{RightCode}
\end{minipage}%
\hspace{\cbdistance}
\begin{minipage}[t]{\cbwidth}
\begin{ErrorCode}
struct Serial_device { ... };
struct SerialDevice { ... };
enum Serial_start_bit { ... };
\end{ErrorCode}
\end{minipage}

\subsection{Variabili}\label{ssec:evariable}

\subsubsection{Variabili ordinarie e membro di strutture}\label{sssec:evarnorm}

Per distinguere il nome delle variabili da quello dei tipi di dato, nelle variabili la prima lettera deve essere minuscola.

\subsubsection{Variabili globali}

Per quanto riguarda le variabili globali queste seguono le stesse regole delle variabili ordinarie.
Anche le variabili globali, come i tipi di dati, se definite all'interno di un modulo devono avere come prefisso il nome del modulo seguito da un underscore.
Se la variabile è locale al modulo (definita nel file .c, esterna a tutte le funzioni) questa deve assolutamente essere definita \texttt{static}, in modo tale da renderla invisibile agli altri moduli.

\noindent\begin{minipage}[t]{\rbwidth}
\begin{RightCode}
static uint8_t Serial_rxBufferHead = 0;
\end{RightCode}
\end{minipage}

\subsection{Funzioni}

Come per le variabili, anche le funzioni devono seguire la notazione \emph{Camel Case}, iniziando con la lettera minuscola\cite{codestyle:geotechnical}.
Il nome della funzione anche in questo caso va preceduta dal nome del modulo e da un underscore.
L'uso dell'underscore non è consentito in nessun altro caso.

\noindent\begin{minipage}[t]{\cbwidth}
\begin{RightCode}
Serial_setBaudrate(uint32_t baud);
Serial_getBaudrate();
\end{RightCode}
\end{minipage}%
\hspace{\cbdistance}
\begin{minipage}[t]{\cbwidth}
\begin{ErrorCode}
Serial_set_baudrate(uint32_t baud);
Serial_GetBaudrate();
\end{ErrorCode}
\end{minipage}

Come per le variabili, anche per le funzioni, se queste sono private e relative solo al modulo è bene che vengano definite \texttt{static}, in modo tale che non siano utilizzabili da nessun altro modulo.

\subsection{Costanti e Macro}\label{ssec:econstantname}

I nomi delle costanti e delle macro, anche in questo caso, devono essere scritte interamente con lettere maiuscole e con le parole separate da un underscore\cite{codestyle:geotechnical}.
Questa regola è valida anche per i valori degli \texttt{enum}.
Nel caso in cui queste costanti siano definite all'interno del modulo, i nomi vanno preceduti dal nome del modulo scritto tutto in maiuscolo e separato dal resto del nome con un underscore.

\noindent\begin{minipage}[t]{\cbwidth}
\begin{RightCode}
const int SERIAL_MAX_DIMENSION = 10;

#define SERIAL_NUM_ITERATIONS  = 2
\end{RightCode}
\end{minipage}%
\hspace{\cbdistance}
\begin{minipage}[t]{\cbwidth}
\begin{ErrorCode}
const int Serial_maxDimension = 10;

#define SERIAL_NUMITERATIONS  = 2
\end{ErrorCode}
\end{minipage}

\subsection{Enumerators}\label{ssec:aenumname}

I valori degli \texttt{enum}, come già detto, seguono le regole delle costanti e delle macro.
Un'ulteriore regola da utilizzare per incrementare la leggibilità del codice prevede l'aggiunta come prefisso ai nomi dei valori sia il nome del modulo e sia il nome del tipo separati da un'underscore.
Nel caso il tipo sia composto da più parole, queste vanno messe tutte in maiuscolo e \emph{non} divise dall'underscore.

\noindent\begin{minipage}[t]{\rbwidth}
\begin{RightCode}
enum Serial_InterruptType
{
    SERIAL_INTERRUPTTYPE_NONE,
    SERIAL_INTERRUPTTYPE_RX,
    SERIAL_INTERRUPTTYPE_TX,
};
\end{RightCode}
\end{minipage}