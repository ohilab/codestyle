\section{Sviluppo delle librerie}

\subsection{Formattazione del codice}

Nella definizione delle funzioni riportare il tipo di dato di ritorno
su una riga e il nome della funzione e i suoi parametri all'inizio
della riga successiva. Questo aiuta gli altri sviluppatori a trovare
dove comincia una nuova funzione.

\smallskip
Le parentesi graffe che contengono l'implementazione della funzione
devono essere posizionate nella prima colonna di una nuova riga di
codice. Infatti molti tool utili a indicizzare le funzioni presenti in
un file cercano proprio nella prima colonna la parentesi graffa di
apertura per localizzare la loro implementazione. Evitare quindi di
inserire altre parentesi di apertura nella prima colonna all'interno
della definizione di una funzione in modo tale da non ingannare tali
strumenti.

La definizione di una funzione assumerà quindi un aspetto simile al
seguente:
\begin{lstlisting}
static char *
concat (char *s1, char *s2)     /* Nome della funzione */
{                       /* Parentesi graffa di apertura */
  ...
}
\end{lstlisting}
Nel caso in cui gli argomenti della funzione siano pittosto numerosi,
suddividerli su più righe in questo modo:
\begin{lstlisting}
int
lots_of_args (int an_integer, long a_long, short a_short,
              double a_double, float a_float)
...
\end{lstlisting}

\smallskip
Come per le funzioni, anche nella dichiarazione di tipi \texttt{struct}
o \texttt{enum} va eseguita mettendo la parentesi graffa di apertura
nella prima colonna, a meno che l'intera dichiarazione non trovi
spazio su un'unica riga:
\begin{lstlisting}
struct foo
{
  int a, b;
}
\end{lstlisting}
oppure
\begin{lstlisting}
struct foo { int a, b; }
\end{lstlisting}

Per quanto riguarda la formattazione del codice all'interno del corpo
delle funzioni lo stile consigliato avrà un'aspetto del tipo:
\begin{lstlisting}
  if (x < foo (y, z))
    haha = bar[4] + 5;
  else
    {
      while (z)
        {
          haha += foo (z, z);
          z--;
        }
      return ++x + bar ();
    }
\end{lstlisting}
In particolare notare lo spazio presente \emph{prima} dell'apertura
delle parentesi (di qualsiasi tipo) e \emph{dopo} le virgole. 

\smallskip
Formattare i cicli \texttt{do-while} nel modo seguente:
\begin{lstlisting}
  do
    {
      a = foo (a);
    }
  while (a > 0);
\end{lstlisting}

\smallskip
Nel caso di  statement \texttt{if-else} annidati all'interno di un
altro \texttt{if}, racchiudere sempre l'\texttt{if-else} all'interno
di parentesi. Quindi, mai scrivere in questo modo:
\begin{lstlisting}
  if (foo)
    if (bar)
      win ();
    else
      lose ();
\end{lstlisting}
ma piuttosto scrivere:
\begin{lstlisting}
  if (foo)
    {
      if (bar)
        win ();
      else
        lose ();
    }
\end{lstlisting}

Se è il caso di un \texttt{if} annidato in un~\texttt{else}, è
possibile se scegliere di scrivere \texttt{else~if} come nel seguente
esempio:
\begin{lstlisting}
  if (bar)
    ...
  else if (foo)
    ...
\end{lstlisting}
con le istruzioni relative all'\texttt{else-if} indentate come quelle
dell'\texttt{if} precedente, oppure includere l'\texttt{if} annidato
tra parentesi:
\begin{lstlisting}
  if (bar)
    ...
  else 
    {
      if (foo)
        ... 
    }
\end{lstlisting}

\smallskip
Evitare di assegnare il valore a delle variabili all'interno della
condizione di un \texttt{if} (all'interno della condizione di un
\texttt{while} invece va bene\ldots). Non scrivere:
\begin{lstlisting}
  if ((foo = (char *) malloc (sizeof *foo)) == 0)
    fatal ("virtual memory exhausted");
\end{lstlisting}
ma piuttosto:
\begin{lstlisting}
  foo = (char *) malloc (sizeof *foo);
  if (foo == 0)
    fatal ("virtual memory exhausted");
\end{lstlisting}

\smallskip
Se dovesse essere necessario suddividere un'espressione su più righe,
è bene andare a capo prima di un operatore e non dopo. Ad esempio in
questo modo:
\begin{lstlisting}
  if (foo_this_is_long && bar > win (x, y, z)
      && remaining_condition)
\end{lstlisting}





\bigskip
Per qualsiasi altra regola sulla formattazione del codice che non è
stata qui specificata fare riferimento allo
\href{http://www.gnu.org/}{standard~GNU}. Nel caso in cui neanche a
questo sito si trovi una risposta, si consiglia comunque di adottare
uno stile e di utilizzarlo con consistenza all'interno di tutto il
progetto.


\subsection{Nomi di variabili, funzioni e files}

La dichiarazione delle funzioni (sia di tipo \texttt{extern} che quelle
che verranno implementate nello stesso file sorgente) deve avvenire
all'inizio del file, prima della definizione della prima funzione. Non
inserire dichiarzione di funzioni \texttt{extern} all'interno di altre
funzioni.

Generalmente è pratica comune utilizzare la stessa variabile locale
(con nomi tipo \texttt{tmp}) per scopi differenti all'interno della
stessa funzione. Anzichè fare ciò, è meglio utilizzare variabili
locali differenti per ogni scopo assegnadogli nomi
significativi. Questo aiuta a comprendere meglio il programma, lasciando
al compilatore il compito di ottimizzare il codice. Se poi si cerca di
posizionare la dichiarazione della varibile locale all'interno dello
\emph{scope} più piccolo che comprende il suo utilizzo rende il codice
ulteriormente più chiaro.

Non utilizzare nomi di variabili o parametri locali che ``oscurano''
variabili globali. L'opzione \texttt{-Wshadow} permette di rilevare
questo tipo di problema.

Non dichiarare strutture e variabili o \texttt{typedef} nella stessa
dichiarazione. Invece dichiarare la struttura e il suo contenuto
separatamente, e poi utilizzala per dichiarare varibili e
\texttt{typedef}.


\bigskip
I nomi di variabili globali e di funzioni deve essere scelto in modo
da fornire anche un'informazione riguardo il loro significato. La
lingua adottata è l'inglese.

Per le variabili locali invece può essere scelto un nome più corto,
poichè il loro utilizzo è limitato all'interno del loro \emph{scope},
dove in genere un commento descrive il loro utilizzo. 

Ad ogni modo cercare di limitare il numero di abbreviazioni nei nomi
delle variabili, ma piuttosto cercare di farne poche, spiegare il loro
significato, e utilizzarle di frequente.


\bigskip
Utilizzare il carattere \emph{underscore}~(\texttt{\_}) per separare le parole nei
nomi di variabili e funzioni (o le maiuscole?).



\subsection{Commentare il codice}

\subsection{Generazione automatica della documentazione}
